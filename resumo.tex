\documentclass[11pt,a4paper]{article}
\usepackage[portuguese]{babel}
\usepackage[utf8]{inputenc}
\usepackage{amsmath}
\usepackage{amsfonts}
\usepackage{enumitem}
\usepackage{booktabs}
\usepackage{circuitikz}
\begin{document}
\section{Linhas de Transmissão}
\subsection{Modelagem da Linha de Transmissão}
Para modelarmos uma linha de transmissão, podemos representá-la com um modelo simples utilizando modelos concentrados, em que representam a linha como 
uma junção de vários circuitos em série em um comprimento $\Delta Z$, assim como mostra a figura Figura~\ref{circ:1}.
\begin{figure}[htpb]
  \centering
  \begin{circuitikz}
    \draw(0,0)
    to[short] (1,0)
    to[R=$G\Delta z$](1,-3)
    to[short] (0,-3);
    \draw(1,0)
    to[short](2,0)
    node[label={[font=\footnotesize]above:A}]{}
    to[short,*-](3,0)
    to[C=$C \Delta z$](3,-3)
    to[short,-*](2,-3)
    to[short](1,-3);
    \draw(3,0)
    to[R=$R \Delta z$,i>^=$i_{(z,t)}$](5,0)
    to[short](5.5,0)
    to[L=$L \Delta z$](6.5,0)
    to[short](7,0)
    to[R=$G \Delta z$,i>^=$i_{G}$](7,-3)
    to[short](2,-3);
    \draw(6.5,0)
    to[short](7,0)
    to[short,-*](8,0)
    node[label={[font=\footnotesize]above:B}]{}
    to[short](9,0)
    to[short,i>^=$i_{z+\Delta z,t}$](10,0)
    ;
    \draw(9,0)
    to[C=$C \Delta z$,i>^=$i_{c}$](9,-3)
    to[short,-*](8,-3)
    to[short](7,-3) 
    ;
    \draw(9,-3)
    to[short](10,-3);
  \end{circuitikz}

  \caption{Circuito equivalente a linha de transmissão no pedaço de linha de comprimento $\Delta z$ com $R \Delta z$ representado as perdas do condutor em Ohms,  a condutância $G \Delta z$ representado as perdas do dielétrico em siemens, a indutância $L \Delta z$ do condutor em henrys e a capacitância $C \Delta z$ em farads.}
  \label{circ:1}
\end{figure}

Aplicando a lei de kirchoff das tensões no trecho $\Delta z$, obteremos:
\begin{align*}
  v(z,t)=i(z,t)R\Delta Z+\frac{\partial i (z,t)}{\partial t}L\Delta z+v(z+\Delta z,t)
\end{align*}
Dividino por $\Delta z$ e rearrajando:
\begin{align*}
  -  \frac{ v(z+\Delta,t)-v(z,t) }{\Delta z} = Ri(z,t)+L \frac{\partial i(z,t)}{\partial t}
\end{align*}
Mas sabemos que:
\begin{align*}
  \lim_{\Delta z \to 0}  \frac{ v(z+\Delta,t)-v(z,t) }{\Delta z}=\frac{\partial v(z,t)}{\partial z}
\end{align*}
Logo:
\begin{align}
  \label{eq:1}
  -\frac{\partial v(z,t)}{\partial z}=Ri(z,t)+L\frac{\partial i(z,t)}{\partial t}
\end{align}
\begin{figure}[hptb]
  \centering
  \begin{circuitikz}
    \draw(0,0)
    to[short,i>^=$i_{z,t}$,-*](2,0)
    node[label={[font=\footnotesize]above:B}]{}
    to[short,i>^=$i_{z+\Delta z,t}$](4,0)
    ;
    \draw(2,0)
    to[short,i>=$i_{G}_{(z+\Delta z,t)}+i_{C}_{(z+ \Delta z,t)}$](2,-2)
    ;
  \end{circuitikz}
  \caption{Correntes entrando e saindo o nó \emph{B}}
  \label{circ:2}
\end{figure}
Vamos agora determinar a outra equação equivalente a Equação~\ref{eq:1}, fazendo uso da equação de Kirchhoff nos nós da ponto \emph{B}.

\begin{align*}
  i(z,t) =i_{G}(z+\Delta z,t)+i_{C}(z+\Delta z,t)+i(z+\Delta z,t)
\end{align*}
Mas sabemos que:
\begin{align*}
  i_{G} (z+\Delta Z,t) &=v(z+ \Delta z,t)G \Delta Z \\
  i_{C}(z+ \Delta z,t) &= \frac{\partial v(z+ \Delta z,t)}{\partial t}C \Delta z
\end{align*}

Que substituindo resulta em:
\begin{align*}
  i(z,t) &= v(z+\Delta z, t)G\Delta z + \frac{\partial v(z+\Delta z,t)}{\partial t} C \Delta z + i(z+ \Delta z, t)
\end{align*}
Reescrevendo e divindo por $\Delta z$, 
\begin{align*}
  -\frac{i(z+\Delta z,t)-i(z,t)}{\Delta z} &= v(z+\Delta z, t)G + \frac{\partial v(z+\Delta z, t)}{\partial t}C
\end{align*}
Observar que:
\begin{equation*}
  \lim_{ \Delta z \to 0} \left[ \frac{i(z+\Delta z,t) -i(z,t)}{\Delta z} \right] = \frac{ \partial i(z,t)}{\partial z}
\end{equation*}
Se temos que,
\begin{equation*}
  v(z+ \Delta z)  = v(z)+ \frac{\partial v (z)}{\partial z}\Delta z
\end{equation*}
Podemos concluir que:
\begin{align*}
  \frac{i(z+\Delta z,t) -i(z,t)}{\Delta z} &= v(z+\Delta z, t)G + \frac{\partial v(z+\Delta z,t)}{\partial t}C\\
  &= G \left[v(z,t) + \frac{\partial v(z,t)}{\partial z} \Delta z \right]+ C \left[ \frac{\partial v(z,t)}{\partial t} + \frac{\partial^2 v(z,t)}{\partial t \partial z}\Delta z \right] \\
  &= \left( G+C \frac{\partial}{\partial t}\right) \left[ v(z,t)+ \frac{ \partial v (z,t) }{\partial z} \Delta z \right]
\end{align*}
Que para $\Delta z \to 0$ se torna:
\begin{align}
  \label{eq:2}
  -\frac{\partial i(z,t)}{\partial z} = Gv(z,t)+ C \frac{\partial v(z,t)}{\partial t}
\end{align}
Dessa forma obtivemos as duas equações fundamentais da linha de transmissão:
\begin{align*}
  -\frac{\partial v(z,t)}{\partial z} &= Ri(z,t)+L\frac{\partial i(z,t)}{\partial t} \tag{\ref{eq:1}}\\
  -\frac{\partial i(z,t)}{\partial z} &= Gv(z,t)+ C \frac{\partial v(z,t)}{\partial t} \tag{\ref{eq:2}}
\end{align*}
Se derivamarmos (\ref{eq:1}) em relação a $z$ e (\ref{eq:2}) em relação a $t$,
\begin{align}
  -\frac{\partial^2 v(z,t)}{\partial z^2} &= R\frac{\partial i(z,t)}{\partial z}+L\frac{\partial^2 i(z,t)}{\partial t \partial z} \label{eq:3}\\
  -\frac{\partial^2 i(z,t)}{\partial t \partial z} &= G\frac{\partial v(z,t)}{\partial t}+ C \frac{\partial^2 v(z,t)}{\partial t^2} \label{eq:4}
\end{align}
Substituindo (\ref{eq:4}) em (\ref{eq:3}):
\begin{equation*}
  \frac{\partial^2 v(z,t)}{\partial z^2} = R \frac{\partial i(z,t)}{\partial z} \frac{\partial v(z,t)}{\partial z} - LC \frac{\partial^2 v(z,t)}{\partial t^2}
\end{equation*}
Substituindo a equação (\ref{eq:2}) acima:
\begin{align*}
  - \frac{\partial^2 v(z,t)}{\partial  z^2} = R \left[ -Gv(z,t) -C \frac{\partial v(z,t)}{\partial t}  \right]- LG \frac{\partial v(z,t)}{\partial t} - LC \frac{\partial^2 v(z,t)}{\partial t^2}
\end{align*}
Assim temos a equação de onda sem perdas para a tensão:
\begin{equation*}
  \frac{\partial^2 v(z,t)}{\partial  z^2}= RGv(z,t) + (RC+LG) \frac{\partial v(z,t)}{\partial t} + LC \frac{\partial^2 v(z,t)}{\partial t^2}
\end{equation*}
E de forma análoga ao procedimento anterior, obtemos a equação de onda para a corrente:
\begin{equation*}
  \frac{\partial^2 i(z,t)}{\partial z^2}  = RGi(z,t) + (RC+LG) \frac{\partial i(z,t)}{\partial t}+ LC \frac{\partial^2 i(z,t)}{\partial t^2}
\end{equation*}
Para linhas sem perdas, temos $R=G=0$ o que nos traz:
\begin{align*}
  \frac{\partial^2 v(z,t)}{\partial  z^2}= LC \frac{\partial^2 v(z,t)}{\partial t^2} \\
  \frac{\partial^2 i(z,t)}{\partial z^2}= LC \frac{\partial ^2 i(z,t)}{\partial t^2}
\end{align*}
\subsection{Soluções para a Equação de Onda}
Nossa onda eletromagnética propaga em ambos os sentidos, logo, a função da tensão deve ser composta por duas funções, uma que caminha na direção positiva de $z$ e outra em sua direção negativa:
\begin{equation*}
  v(z,t) = v^+ \left[ t- \frac{z}{v_f}\right]+ v^- \left[ t+ \frac{z}{v_f}\right]
\end{equation*}
Onde:
\begin{align*}
  v^+ \left[ t- \frac{z}{v_f}\right] &\text{ representa a onda propagando na direção positiva de $z$}\\
  v^- \left[ t+ \frac{z}{v_f}\right]  &\text{ representa a onda propagando na direção negativa de $z$ }\\
  v_{f}=\frac{1}{\sqrt{LC}} &\text{ representa a velocidade de propagação da onda (a velocidade de fase)}
\end{align*}
Assim é possível mostrar que as soluções abaixo são soluções das equações de onda para tensão, em linhas de transmissão sem perdas:
\begin{equation*}
  v(z,t)= V^+_0 \cos \left[ \omega  \left( t- \frac{z}{v_f} \right) +\phi^+  \right]+ V^-_0 \cos \left[  \omega  \left( t+ \frac{z}{v_f}  \right) + \phi^+  \right]
\end{equation*}
Se adotarmos uma constante de propagação, $k_{z}$, como $k_{z}=\frac{ \omega }{v_{f}}$, obteremos:
\begin{equation*}
  v(z,t)= V^+_0 \cos \left[ \omega  \left( t- \frac{z}{v_f} \right) +\phi^+  \right]+ V^-_0 \cos \left[  \omega  \left( t+ \frac{z}{v_f}  \right) + \phi^+  \right]
\end{equation*}
Como sabemos que a frequência angular e amplitude são invariantes no domínio do tempo, podemos lembrar da fórmula de Euler:
\begin{equation*}
  e^{j \zeta} = \cos(\zeta)+ j \sin(\zeta)
\end{equation*}
E escrever:
\begin{equation*}
  A \cos( \omega t+ \phi) = Re \left\{ A e^{ \left[ j( \omega t+ \phi) \right] } \right\} Re \left\{ A e^{j \phi} e^{j \omega t} \right\}
\end{equation*}
Onde,
\begin{align*}
  A e^{\left(j \phi\right)}e^{ \left( j \omega t \right)} : &\text{ a função complexa instantanêa} \\
  A e^{\left(j \phi\right)} : &\text{ a função fasorial}
\end{align*}

Assim, podemos nos recordamos do conceito de fasor, usado em circuitos elétricos, com o exemplo abaixo:
\begin{align*}
  v(t)=V_0 cos( \omega t) \\
  v(t)= v_1 (t) + v_2 (t)
\end{align*}
\begin{figure}[htpb]
  \centering
  \begin{circuitikz}
    \draw (0,0)
    to[sV, v=$v(t)$](0,2)
    to[short](1,2);
    \draw (1,2) to[R,v^<=$v_1$] (2,2);
    \draw(2,2)
    to[short](3,2)
    to[R,v^<=$v_2$](4,2)
    to[short](5,2)
    to[short](5,0)
    to[short](0,0);
  \end{circuitikz}
  \caption{Circuito Exemplo}
  \label{circ:3}
\end{figure}
Pela lei de Ohm podemos equacionar:
\begin{equation*}
  v(t)= R_1 i(t)+ R_2 i(t)= (R_1 + R_2 )i(t)
\end{equation*}
\begin{equation*}V_0 \cos (\omega t)= (R_1 + R_2)i(t)\end{equation*}
\begin{equation*}  i(t)=\frac{V_0}{R_1 R_2}\cos(\omega t)\end{equation*}
\begin{equation*}
  I= \frac{V_0}{R_1+R_2}\mathbf{fasor}
\end{equation*}
Como temos:
\begin{align*}
  v^+ (z,t) &= V_0^+ \cos( \omega t -kz + \phi^+ ) \\
  v^+ (z,t) &= Re \left\{V_0^+ \left[\cos( \omega t -kz + \phi^+ )+ j \sin (wt-kz + \phi^+) \right]   \right\} \\
  v^+ (z,t) &= Re \left\{ V_o^+ e^{ \left[ j \left( \omega - kz + \phi^+ \right) \right]} \right\}\\
  v^+ (z,t) &= Re \left\{ V_o^+ e^{\phi^+}  e^{-jkz}  e^{j \omega t} \right\}
\end{align*}
Temos a tensão instantanêa real como:
\begin{equation*}
  \mathcal{V}(z,t) = V_0 \cos (\omega t - kz + \phi)
\end{equation*}
Tensão complexa instantanêa:
\begin{equation*}
  V(z,t)= V_0 e^{j \phi} e^{-jkz} e^{jwt}
\end{equation*}
E a tensão fasorial:
\begin{equation*}
  V(z)  = V_0 e^{j \phi} e^ {-jkz}
\end{equation*}
Lembremos então da equação de onda e excitação senoidal:
\begin{align*}  
  v^+ = (z,t) &= Re \left\{ A e^{ \left[ j( \omega t+ \phi) \right] } \right\} Re \left\{ A e^{j \phi} e^{j \omega t} \right\}\\
  &= Re \left\{ V^+(z) e^{j \omega t} \right\}
\end{align*}
A derivada temporal se tornará então:
\begin{align*}
  \frac{\partial}{\partial t}e^{j \omega t} = jw e^{j \omega t}
\end{align*}
E a derivada espacial:
\begin{align*}
  \frac{\partial}{\partial z} e^{-jk_z z} = -jk_z e^{-j k_z z}
\end{align*}
Ou seja, uma derivada temporal no domínio da frequência equivale a:
\begin{align*}
  \frac{\partial}{\partial t} \Leftrightarrow j \omega
\end{align*}
E uma derivada espacial equivale a:
\begin{align*}
  \frac{\partial}{\partial z} \Leftrightarrow -j k_z
\end{align*}
Com esse conhecimento, podemos transformar nossa equação de onda para a tensão:
\begin{align*}
  \frac{\partial^2 v(z,t) }{\partial z^2} = RG v(z,t) + (RC+LG) \frac{\partial v(z,t)}{\partial t} + LC \frac{\partial^2 v(z,t)}{\partial t^2}
\end{align*}
Em:
\begin{align}
  \frac{\partial^2 V(z)}{\partial z^2} = \left[ \left( RG - \omega^2 LC \right) + j \omega (RC+LG)  \right]V(Z) \label{eq:5}
\end{align}
E analogamente, a de corrente:
\begin{align*}
  \frac{\partial^2 i(z,t)}{\partial z^2} = RGi(z,t) + (RC+LG) \frac{\partial i(z,t)}{\partial t}+ LC \frac{\partial^2 i(z,t)}{\partial t^2}
\end{align*}
Em:
\begin{align}
  \frac{\partial^2 I(Z)}{\partial z^2}= \left[ \left( RG - \omega^2 LC \right) + j\omega (RC+LG) \right]I(z) \label{eq:6}
\end{align}
\subsection{Impedância e Admitância}
\begin{align*}
  \frac{\partial^2 v(z,t) }{\partial z^2} &= RG v(z,t) + (RC+LG) \frac{\partial v(z,t)}{\partial t} + LC \frac{\partial^2 v(z,t)}{\partial t^2} \quad \quad \text{Tensão}\\
  \frac{\partial^2 i(z,t)}{\partial z^2} &= RGi(z,t) + (RC+LG) \frac{\partial i(z,t)}{\partial t}+ LC \frac{\partial^2 i(z,t)}{\partial t^2} \quad \text{Corrente}
\end{align*}
Se temos:
\begin{align*}
  Z &= R+ j\omega L \quad \text{Impedância ($\Omega /m$ - Ohms$/m$)} \\
  Y &= G+ j \omega C \quad \text{Admitância ($S/m$ - Siemens$/m$)} 
\end{align*}
Então,
\begin{align*}
  ZY = (RG - \omega^2 LC) + j \omega (RC+LG)
\end{align*}
Onde:
\begin{align*}
  Z= R+j\omega L \quad \text{Impedância (por u.c.)} \\
  Y= G+j\omega C \quad \text{Admitância (por u.c.)} 
\end{align*}
\subsection{Constante de Propagação}
A constante de propagação é dada por:
\begin{align*}
  \gamma = \sqrt{ZY} = \alpha + j \beta
\end{align*}
Em que,
\begin{align*}
  \alpha  \to & \quad \text{constante de atenuação; neper/unidade de comprimento} \\
  \beta \to & \quad \text{constante de fase; radiano/unidade de comprimento} \\
\end{align*}
Em uma linha sem perdas, temos:
\begin{align*}
  ZY &= -\omega^2 LC\\
  \gamma = \sqrt{ZY} &=j \beta = j \omega \sqrt{LC}\\
  \alpha &= 0
\end{align*}
Podemos então relacionar,
\begin{align*}
  \gamma &= \sqrt{ZY} \\
   \gamma^2 &= ZY \\
   ZY = (RG - \omega^2 LC)+ j \omega (RC + LG)
\end{align*}
E substituir em:
\begin{align*}
  \frac{\partial^2 V(z)}{\partial z^2} = \left[ (RG - \omega^2 LC) + j \omega (RC+LG) \right]V(z) 
\end{align*}
Transformando em:
\begin{align*}
\frac{\partial^2 V(z)}{\partial z^2}= \gamma^2 V(z)
\end{align*}
E de forma análoga para a corrente:
\begin{align*}
  \frac{\partial^2 I(z)}{\partial z^2} &= \left[ (RG - \omega^2 LC) + j \omega (RC+LG) \right]I(z) \\
\frac{\partial^2 I(z)}{\partial z^2} &= \gamma^2 I(z)
\end{align*}
Podemos ainda relacionar a impedância com as equações (\ref{eq:5}) e (\ref{eq:6}),
\begin{align*}
  \begin{cases}
    - \frac{\partial v(z,t)}{\partial z}  = Ri(z,t) + L \frac{\partial i(z,t)}{\partial t} \\
    - \frac{\partial i(z,t)}{\partial  z} = G v(z,t) + C \frac{\partial v(z,t)}{\partial t}
  \end{cases}   
\end{align*}
Sabemos que $ \frac{\partial }{\partial t} e^{j \omega t}= j \omega e^{j \omega t}$:
\begin{align*}
  \begin{cases}
   -\frac{\partial V(z)}{\partial z}  = RI (z) + j \omega LI (z) = (R + j \omega L) I(Z) \\
   - \frac{\partial I(z)}{\partial z} = GC (z) + j \omega CV (z) = (G + j \omega C) V(z)
  \end{cases}
\end{align*}
O que nos dá:
\begin{align*}
  \frac{\partial V(z)}{\partial z} = - ZI(z) \\
  \frac{\partial I(z)}{\partial z} = -YV(z)
\end{align*}
\subsection{Resumo das equações fundamentais dee uma L.T}
\begin{table}[htpb]
  \centering
  \label{tab:1}
  \begin{tabular}{c c c}
    \toprule
    &  V(z)& I(z)  \\ \midrule
       $\frac{\partial}{\partial z}$ & $-YV(z)$ & $-ZI(z)$\\ \midrule
       $\frac{\partial^2 }{\partial z^2 }$ & $\gamma^2 V(z)$ & $\gamma^2 I(z)$ \\ \bottomrule
  \end{tabular}
  \caption{Resumo das equações fundamentais}
\end{table}

Em que:
  \begin{align*}
  Z &= R + j \omega L \\
  Y &= G + j \omega C \\
  ZY &= (RG- \omega^2 LC) + j \omega (RC+LG) \\
  \gamma^2 &= ZY
  \end{align*}
  \subsection{Soluções das equações de onda}
  Assim definimos que a solução de onda para a tensão:
  \begin{align*}
    \frac{\partial^2 V(z)}{\partial z^2} - \gamma^2 V(z) =0 
  \end{align*} 
  É da forma:
  \begin{align*}
  V(z)= V^+ e^{- \gamma z} + V^- e^{+\gamma z}
  \end{align*}
  Em que $V^+$ e $V^-$ são constantes a serem determinadas.
  
  Se lembrarmos que:
  \begin{align*}
    \frac{\partial V(z)}{\partial z} = -Z I(z)
  \end{align*}
  Podemos escrever:
  \begin{align*}
    I(z) = \frac{\gamma}{Z} (V^+ e^{-\gamma z} - V^- e^{-\gamma z})
  \end{align*}
  Em que $I^+ = \frac{\gamma }{Z} V^+$ e $I^- = \frac{\gamma}{Z}V^-$ também são constantes a determinadas, pela equação da corrente ou através das constantes $V^+$ e $V^-$. Ainda podemos fazer uma análise dimensional em:
  \begin{align*}
    \frac{\gamma}{Z} = \frac{\sqrt{ZY}}{Z}= \sqrt{\frac{ZY}{Z^2}}= \sqrt{\frac{Y}{Z}} 
  \end{align*}
 O que nos dá:
 \begin{align*}
   \left[ \frac{\gamma}{Z} \right]= \text{ohm}^{-1}
 \end{align*}
 Que é o que esperamos pela Lei de Ohm.

 Em resumo temos:
 \begin{align*}
 V(z)= V^+ e^{- \gamma z}  + V^- e^{\gamma z} \\
I(z)= I^+ e^{- \gamma z}  - I^- e^{\gamma z} \\
 \end{align*}
\end{document}
