\documentclass[11pt,a4paper]{article}
\usepackage[portuguese]{babel}
\usepackage[utf8]{inputenc}
\usepackage{amsmath}
\usepackage{amssymb}
\usepackage{amsfonts}
\usepackage{enumitem}
\usepackage{booktabs}
\usepackage{circuitikz}
% TODO Trocar alguns \tau_0  por \Gamma
\begin{document}
\section{Linhas de Transmissão}
\subsection{Modelagem da Linha de Transmissão}
Para modelarmos uma linha de transmissão, podemos representá-la com um modelo simples utilizando modelos concentrados, em que representam a linha como 
uma junção de vários circuitos em série em um comprimento $\Delta Z$, assim como mostra a figura Figura~\ref{circ:1}.
\begin{figure}[htpb]
  \centering
  \begin{circuitikz}
    \draw(0,0)
    to[short] (1,0)
    to[R=$G\Delta z$](1,-3)
    to[short] (0,-3);
    \draw(1,0)
    to[short](2,0)
    node[label={[font=\footnotesize]above:A}]{}
    to[short,*-](3,0)
    to[C=$C \Delta z$](3,-3)
    to[short,-*](2,-3)
    to[short](1,-3);
    \draw(3,0)
    to[R=$R \Delta z$,i>^=$i_{(z,t)}$](5,0)
    to[short](5.5,0)
    to[L=$L \Delta z$](6.5,0)
    to[short](7,0)
    to[R=$G \Delta z$,i>^=$i_{G}$](7,-3)
    to[short](2,-3);
    \draw(6.5,0)
    to[short](7,0)
    to[short,-*](8,0)
    node[label={[font=\footnotesize]above:B}]{}
    to[short](9,0)
    to[short,i>^=$i_{z+\Delta z,t}$](10,0)
    ;
    \draw(9,0)
    to[C=$C \Delta z$,i>^=$i_{c}$](9,-3)
    to[short,-*](8,-3)
    to[short](7,-3) 
    ;
    \draw(9,-3)
    to[short](10,-3);
  \end{circuitikz}

  \caption{Circuito equivalente a linha de transmissão no pedaço de linha de comprimento $\Delta z$ com $R \Delta z$ representado as perdas do condutor em Ohms,  a condutância $G \Delta z$ representado as perdas do dielétrico em siemens, a indutância $L \Delta z$ do condutor em henrys e a capacitância $C \Delta z$ em farads.}
  \label{circ:1}
\end{figure}

Aplicando a lei de kirchoff das tensões no trecho $\Delta z$, obteremos:
\begin{align*}
  v(z,t)=i(z,t)R\Delta Z+\frac{\partial i (z,t)}{\partial t}L\Delta z+v(z+\Delta z,t)
\end{align*}
Dividino por $\Delta z$ e rearrajando:
\begin{align*}
  -  \frac{ v(z+\Delta,t)-v(z,t) }{\Delta z} = Ri(z,t)+L \frac{\partial i(z,t)}{\partial t}
\end{align*}
Mas sabemos que:
\begin{align*}
  \lim_{\Delta z \to 0}  \frac{ v(z+\Delta,t)-v(z,t) }{\Delta z}=\frac{\partial v(z,t)}{\partial z}
\end{align*}
Logo:
\begin{align}
  \label{eq:1}
  -\frac{\partial v(z,t)}{\partial z}=Ri(z,t)+L\frac{\partial i(z,t)}{\partial t}
\end{align}
\begin{figure}[hptb]
  \centering
  \begin{circuitikz}
    \draw(0,0)
    to[short,i>^=$i_{z,t}$,-*](2,0)
    node[label={[font=\footnotesize]above:B}]{}
    to[short,i>^=$i_{z+\Delta z,t}$](4,0)
    ;
    \draw(2,0)
    to[short,i>=$i_{G}_{(z+\Delta z,t)}+i_{C}_{(z+ \Delta z,t)}$](2,-2)
    ;
  \end{circuitikz}
  \caption{Correntes entrando e saindo o nó \emph{B}}
  \label{circ:2}
\end{figure}
Vamos agora determinar a outra equação equivalente a Equação~\ref{eq:1}, fazendo uso da equação de Kirchhoff nos nós da ponto \emph{B}.

\begin{align*}
  i(z,t) =i_{G}(z+\Delta z,t)+i_{C}(z+\Delta z,t)+i(z+\Delta z,t)
\end{align*}
Mas sabemos que:
\begin{align*}
  i_{G} (z+\Delta Z,t) &=v(z+ \Delta z,t)G \Delta Z \\
  i_{C}(z+ \Delta z,t) &= \frac{\partial v(z+ \Delta z,t)}{\partial t}C \Delta z
\end{align*}

Que substituindo resulta em:
\begin{align*}
  i(z,t) &= v(z+\Delta z, t)G\Delta z + \frac{\partial v(z+\Delta z,t)}{\partial t} C \Delta z + i(z+ \Delta z, t)
\end{align*}
Reescrevendo e divindo por $\Delta z$, 
\begin{align*}
  -\frac{i(z+\Delta z,t)-i(z,t)}{\Delta z} &= v(z+\Delta z, t)G + \frac{\partial v(z+\Delta z, t)}{\partial t}C
\end{align*}
Observar que:
\begin{equation*}
  \lim_{ \Delta z \to 0} \left[ \frac{i(z+\Delta z,t) -i(z,t)}{\Delta z} \right] = \frac{ \partial i(z,t)}{\partial z}
\end{equation*}
Se temos que,
\begin{equation*}
  v(z+ \Delta z)  = v(z)+ \frac{\partial v (z)}{\partial z}\Delta z
\end{equation*}
Podemos concluir que:
\begin{align*}
  \frac{i(z+\Delta z,t) -i(z,t)}{\Delta z} &= v(z+\Delta z, t)G + \frac{\partial v(z+\Delta z,t)}{\partial t}C\\
  &= G \left[v(z,t) + \frac{\partial v(z,t)}{\partial z} \Delta z \right]+ C \left[ \frac{\partial v(z,t)}{\partial t} + \frac{\partial^2 v(z,t)}{\partial t \partial z}\Delta z \right] \\
  &= \left( G+C \frac{\partial}{\partial t}\right) \left[ v(z,t)+ \frac{ \partial v (z,t) }{\partial z} \Delta z \right]
\end{align*}
Que para $\Delta z \to 0$ se torna:
\begin{align}
  \label{eq:2}
  -\frac{\partial i(z,t)}{\partial z} = Gv(z,t)+ C \frac{\partial v(z,t)}{\partial t}
\end{align}
Dessa forma obtivemos as duas equações fundamentais da linha de transmissão:
\begin{align*}
  -\frac{\partial v(z,t)}{\partial z} &= Ri(z,t)+L\frac{\partial i(z,t)}{\partial t} \tag{\ref{eq:1}}\\
  -\frac{\partial i(z,t)}{\partial z} &= Gv(z,t)+ C \frac{\partial v(z,t)}{\partial t} \tag{\ref{eq:2}}
\end{align*}
Se derivamarmos (\ref{eq:1}) em relação a $z$ e (\ref{eq:2}) em relação a $t$,
\begin{align}
  -\frac{\partial^2 v(z,t)}{\partial z^2} &= R\frac{\partial i(z,t)}{\partial z}+L\frac{\partial^2 i(z,t)}{\partial t \partial z} \label{eq:3}\\
  -\frac{\partial^2 i(z,t)}{\partial t \partial z} &= G\frac{\partial v(z,t)}{\partial t}+ C \frac{\partial^2 v(z,t)}{\partial t^2} \label{eq:4}
\end{align}
Substituindo (\ref{eq:4}) em (\ref{eq:3}):
\begin{equation*}
  \frac{\partial^2 v(z,t)}{\partial z^2} = R \frac{\partial i(z,t)}{\partial z} \frac{\partial v(z,t)}{\partial z} - LC \frac{\partial^2 v(z,t)}{\partial t^2}
\end{equation*}
Substituindo a equação (\ref{eq:2}) acima:
\begin{align*}
  - \frac{\partial^2 v(z,t)}{\partial  z^2} = R \left[ -Gv(z,t) -C \frac{\partial v(z,t)}{\partial t}  \right]- LG \frac{\partial v(z,t)}{\partial t} - LC \frac{\partial^2 v(z,t)}{\partial t^2}
\end{align*}
Assim temos a equação de onda sem perdas para a tensão:
\begin{equation*}
  \frac{\partial^2 v(z,t)}{\partial  z^2}= RGv(z,t) + (RC+LG) \frac{\partial v(z,t)}{\partial t} + LC \frac{\partial^2 v(z,t)}{\partial t^2}
\end{equation*}
E de forma análoga ao procedimento anterior, obtemos a equação de onda para a corrente:
\begin{equation*}
  \frac{\partial^2 i(z,t)}{\partial z^2}  = RGi(z,t) + (RC+LG) \frac{\partial i(z,t)}{\partial t}+ LC \frac{\partial^2 i(z,t)}{\partial t^2}
\end{equation*}
Para linhas sem perdas, temos $R=G=0$ o que nos traz:
\begin{align*}
  \frac{\partial^2 v(z,t)}{\partial  z^2}= LC \frac{\partial^2 v(z,t)}{\partial t^2} \\
  \frac{\partial^2 i(z,t)}{\partial z^2}= LC \frac{\partial ^2 i(z,t)}{\partial t^2}
\end{align*}
\subsection{Soluções para a Equação de Onda}
Nossa onda eletromagnética propaga em ambos os sentidos, logo, a função da tensão deve ser composta por duas funções, uma que caminha na direção positiva de $z$ e outra em sua direção negativa:
\begin{equation*}
  v(z,t) = v^+ \left[ t- \frac{z}{v_f}\right]+ v^- \left[ t+ \frac{z}{v_f}\right]
\end{equation*}
Onde:
\begin{align*}
  v^+ \left[ t- \frac{z}{v_f}\right] &\text{ representa a onda propagando na direção positiva de $z$}\\
  v^- \left[ t+ \frac{z}{v_f}\right]  &\text{ representa a onda propagando na direção negativa de $z$ }\\
  v_{f}=\frac{1}{\sqrt{LC}} &\text{ representa a velocidade de propagação da onda (a velocidade de fase)}
\end{align*}
Assim é possível mostrar que as soluções abaixo são soluções das equações de onda para tensão, em linhas de transmissão sem perdas:
\begin{equation*}
  v(z,t)= V^+_0 \cos \left[ \omega  \left( t- \frac{z}{v_f} \right) +\phi^+  \right]+ V^-_0 \cos \left[  \omega  \left( t+ \frac{z}{v_f}  \right) + \phi^+  \right]
\end{equation*}
Se adotarmos uma constante de propagação, $k_{z}$, como $k_{z}=\frac{ \omega }{v_{f}}$, obteremos:
\begin{equation*}
  v(z,t)= V^+_0 \cos \left[ \omega  \left( t- \frac{z}{v_f} \right) +\phi^+  \right]+ V^-_0 \cos \left[  \omega  \left( t+ \frac{z}{v_f}  \right) + \phi^+  \right]
\end{equation*}
Como sabemos que a frequência angular e amplitude são invariantes no domínio do tempo, podemos lembrar da fórmula de Euler:
\begin{equation*}
  e^{j \zeta} = \cos(\zeta)+ j \sin(\zeta)
\end{equation*}
E escrever:
\begin{equation*}
  A \cos( \omega t+ \phi) = Re \left\{ A e^{ \left[ j( \omega t+ \phi) \right] } \right\} Re \left\{ A e^{j \phi} e^{j \omega t} \right\}
\end{equation*}
Onde,
\begin{align*}
  A e^{\left(j \phi\right)}e^{ \left( j \omega t \right)} : &\text{ a função complexa instantanêa} \\
  A e^{\left(j \phi\right)} : &\text{ a função fasorial}
\end{align*}

Assim, podemos nos recordamos do conceito de fasor, usado em circuitos elétricos, com o exemplo abaixo:
\begin{align*}
  v(t)=V_0 cos( \omega t) \\
  v(t)= v_1 (t) + v_2 (t)
\end{align*}
\begin{figure}[htpb]
  \centering
  \begin{circuitikz}
    \draw (0,0)
    to[sV, v=$v(t)$](0,2)
    to[short](1,2);
    \draw (1,2) to[R,v^<=$v_1$] (2,2);
    \draw(2,2)
    to[short](3,2)
    to[R,v^<=$v_2$](4,2)
    to[short](5,2)
    to[short](5,0)
    to[short](0,0);
  \end{circuitikz}
  \caption{Circuito Exemplo}
  \label{circ:3}
\end{figure}
Pela lei de Ohm podemos equacionar:
\begin{equation*}
  v(t)= R_1 i(t)+ R_2 i(t)= (R_1 + R_2 )i(t)
\end{equation*}
\begin{equation*}V_0 \cos (\omega t)= (R_1 + R_2)i(t)\end{equation*}
\begin{equation*}  i(t)=\frac{V_0}{R_1 R_2}\cos(\omega t)\end{equation*}
\begin{equation*}
  I= \frac{V_0}{R_1+R_2}\mathbf{fasor}
\end{equation*}
Como temos:
\begin{align*}
  v^+ (z,t) &= V_0^+ \cos( \omega t -kz + \phi^+ ) \\
  v^+ (z,t) &= Re \left\{V_0^+ \left[\cos( \omega t -kz + \phi^+ )+ j \sin (wt-kz + \phi^+) \right]   \right\} \\
  v^+ (z,t) &= Re \left\{ V_o^+ e^{ \left[ j \left( \omega - kz + \phi^+ \right) \right]} \right\}\\
  v^+ (z,t) &= Re \left\{ V_o^+ e^{\phi^+}  e^{-jkz}  e^{j \omega t} \right\}
\end{align*}
Temos a tensão instantanêa real como:
\begin{equation*}
  \mathcal{V}(z,t) = V_0 \cos (\omega t - kz + \phi)
\end{equation*}
Tensão complexa instantanêa:
\begin{equation*}
  V(z,t)= V_0 e^{j \phi} e^{-jkz} e^{jwt}
\end{equation*}
E a tensão fasorial:
\begin{equation*}
  V(z)  = V_0 e^{j \phi} e^ {-jkz}
\end{equation*}
Lembremos então da equação de onda e excitação senoidal:
\begin{align*}  
  v^+ = (z,t) &= Re \left\{ A e^{ \left[ j( \omega t+ \phi) \right] } \right\} Re \left\{ A e^{j \phi} e^{j \omega t} \right\}\\
  &= Re \left\{ V^+(z) e^{j \omega t} \right\}
\end{align*}
A derivada temporal se tornará então:
\begin{align*}
  \frac{\partial}{\partial t}e^{j \omega t} = jw e^{j \omega t}
\end{align*}
E a derivada espacial:
\begin{align*}
  \frac{\partial}{\partial z} e^{-jk_z z} = -jk_z e^{-j k_z z}
\end{align*}
Ou seja, uma derivada temporal no domínio da frequência equivale a:
\begin{align*}
  \frac{\partial}{\partial t} \Leftrightarrow j \omega
\end{align*}
E uma derivada espacial equivale a:
\begin{align*}
  \frac{\partial}{\partial z} \Leftrightarrow -j k_z
\end{align*}
Com esse conhecimento, podemos transformar nossa equação de onda para a tensão:
\begin{align*}
  \frac{\partial^2 v(z,t) }{\partial z^2} = RG v(z,t) + (RC+LG) \frac{\partial v(z,t)}{\partial t} + LC \frac{\partial^2 v(z,t)}{\partial t^2}
\end{align*}
Em:
\begin{align}
  \frac{\partial^2 V(z)}{\partial z^2} = \left[ \left( RG - \omega^2 LC \right) + j \omega (RC+LG)  \right]V(Z) \label{eq:5}
\end{align}
E analogamente, a de corrente:
\begin{align*}
  \frac{\partial^2 i(z,t)}{\partial z^2} = RGi(z,t) + (RC+LG) \frac{\partial i(z,t)}{\partial t}+ LC \frac{\partial^2 i(z,t)}{\partial t^2}
\end{align*}
Em:
\begin{align}
  \frac{\partial^2 I(Z)}{\partial z^2}= \left[ \left( RG - \omega^2 LC \right) + j\omega (RC+LG) \right]I(z) \label{eq:6}
\end{align}
\subsection{Impedância e Admitância}
\begin{align*}
  \frac{\partial^2 v(z,t) }{\partial z^2} &= RG v(z,t) + (RC+LG) \frac{\partial v(z,t)}{\partial t} + LC \frac{\partial^2 v(z,t)}{\partial t^2} \quad \quad \text{Tensão}\\
  \frac{\partial^2 i(z,t)}{\partial z^2} &= RGi(z,t) + (RC+LG) \frac{\partial i(z,t)}{\partial t}+ LC \frac{\partial^2 i(z,t)}{\partial t^2} \quad \text{Corrente}
\end{align*}
Se temos:
\begin{align*}
  Z &= R+ j\omega L \quad \text{Impedância ($\Omega /m$ - Ohms$/m$)} \\
  Y &= G+ j \omega C \quad \text{Admitância ($S/m$ - Siemens$/m$)} 
\end{align*}
Então,
\begin{align*}
  ZY = (RG - \omega^2 LC) + j \omega (RC+LG)
\end{align*}
Onde:
\begin{align*}
  Z= R+j\omega L \quad \text{Impedância (por u.c.)} \\
  Y= G+j\omega C \quad \text{Admitância (por u.c.)} 
\end{align*}
\subsection{Constante de Propagação}
A constante de propagação é dada por:
\begin{align*}
  \gamma = \sqrt{ZY} = \alpha + j \beta
\end{align*}
Em que,
\begin{align*}
  \alpha  \to & \quad \text{constante de atenuação; neper/unidade de comprimento} \\
  \beta \to & \quad \text{constante de fase; radiano/unidade de comprimento} \\
\end{align*}
Em uma linha sem perdas, temos:
\begin{align*}
  ZY &= -\omega^2 LC\\
  \gamma = \sqrt{ZY} &=j \beta = j \omega \sqrt{LC}\\
  \alpha &= 0
\end{align*}
Podemos então relacionar,
\begin{align*}
  \gamma &= \sqrt{ZY} \\
   \gamma^2 &= ZY \\
   ZY = (RG - \omega^2 LC)+ j \omega (RC + LG)
\end{align*}
E substituir em:
\begin{align*}
  \frac{\partial^2 V(z)}{\partial z^2} = \left[ (RG - \omega^2 LC) + j \omega (RC+LG) \right]V(z) 
\end{align*}
Transformando em:
\begin{align*}
\frac{\partial^2 V(z)}{\partial z^2}= \gamma^2 V(z)
\end{align*}
E de forma análoga para a corrente:
\begin{align*}
  \frac{\partial^2 I(z)}{\partial z^2} &= \left[ (RG - \omega^2 LC) + j \omega (RC+LG) \right]I(z) \\
\frac{\partial^2 I(z)}{\partial z^2} &= \gamma^2 I(z)
\end{align*}
Podemos ainda relacionar a impedância com as equações (\ref{eq:5}) e (\ref{eq:6}),
\begin{align*}
  \begin{cases}
    - \frac{\partial v(z,t)}{\partial z}  = Ri(z,t) + L \frac{\partial i(z,t)}{\partial t} \\
    - \frac{\partial i(z,t)}{\partial  z} = G v(z,t) + C \frac{\partial v(z,t)}{\partial t}
  \end{cases}   
\end{align*}
Sabemos que $ \frac{\partial }{\partial t} e^{j \omega t}= j \omega e^{j \omega t}$:
\begin{align*}
  \begin{cases}
   -\frac{\partial V(z)}{\partial z}  = RI (z) + j \omega LI (z) = (R + j \omega L) I(Z) \\
   - \frac{\partial I(z)}{\partial z} = GC (z) + j \omega CV (z) = (G + j \omega C) V(z)
  \end{cases}
\end{align*}
O que nos dá:
\begin{align*}
  \frac{\partial V(z)}{\partial z} = - ZI(z) \\
  \frac{\partial I(z)}{\partial z} = -YV(z)
\end{align*}
\subsection{Resumo das equações fundamentais dee uma L.T}
\begin{table}[htpb]
  \centering
  \label{tab:1}
  \begin{tabular}{c c c}
    \toprule
    &  V(z)& I(z)  \\ \midrule
       $\frac{\partial}{\partial z}$ & $-YV(z)$ & $-ZI(z)$\\ \midrule
       $\frac{\partial^2 }{\partial z^2 }$ & $\gamma^2 V(z)$ & $\gamma^2 I(z)$ \\ \bottomrule
  \end{tabular}
  \caption{Resumo das equações fundamentais}
\end{table}

Em que:
  \begin{align*}
  Z &= R + j \omega L \\
  Y &= G + j \omega C \\
  ZY &= (RG- \omega^2 LC) + j \omega (RC+LG) \\
  \gamma^2 &= ZY
  \end{align*}
  \subsection{Soluções das equações de onda}
  Assim definimos que a solução de onda para a tensão:
  \begin{align*}
    \frac{\partial^2 V(z)}{\partial z^2} - \gamma^2 V(z) =0 
  \end{align*} 
  É da forma:
  \begin{align*}
  V(z)= V^+ e^{- \gamma z} + V^- e^{+\gamma z}
  \end{align*}
  Em que $V^+$ e $V^-$ são constantes a serem determinadas.
  
  Se lembrarmos que:
  \begin{align*}
    \frac{\partial V(z)}{\partial z} = -Z I(z)
  \end{align*}
  Podemos escrever:
  \begin{align*}
    I(z) = \frac{\gamma}{Z} (V^+ e^{-\gamma z} - V^- e^{-\gamma z})
  \end{align*}
  Em que $I^+ = \frac{\gamma }{Z} V^+$ e $I^- = \frac{\gamma}{Z}V^-$ também são constantes a determinadas, pela equação da corrente ou através das constantes $V^+$ e $V^-$. Ainda podemos fazer uma análise dimensional em:
  \begin{align*}
    \frac{\gamma}{Z} = \frac{\sqrt{ZY}}{Z}= \sqrt{\frac{ZY}{Z^2}}= \sqrt{\frac{Y}{Z}} 
  \end{align*}
 O que nos dá:
 \begin{align*}
   \left[ \frac{\gamma}{Z} \right]= \text{ohm}^{-1}
 \end{align*}
 Que é o que esperamos pela Lei de Ohm.

 Em resumo temos:
 \begin{align*}
 V(z)= V^+ e^{- \gamma z}  + V^- e^{\gamma z} \\
I(z)= I^+ e^{- \gamma z}  - I^- e^{\gamma z} \\
 \end{align*}
 \subsection{Conceito de Perturbação}
 \begin{itemize}
   \item Uma perturbação move-se a partir de uma fonte ao longo do tmepo
   \item O meio pode não apresentar movimento na direção de propagação
   \item Propagação de onda eletromagnética
     \begin{itemize}
       \item Não é necessário meio físico
       \item Há propagação no vácuo
     \end{itemize}
   \item Tipos de ondas
     \begin{itemize}
       \item Ondas Mecânicas
       \item Ondas Eletromagnêticas 
     \end{itemize}
 \end{itemize}
\begin{figure}[htpb]
\centering
\begin{circuitikz}
  \draw(0,0) 
  to[C](0,2)
  to[L](2,2)
  to[C](2,0)
  to[short](0,0);
  \draw(2,2)
  to[L](4,2)
  to[C](4,0)
  to[short](2,0);
  \draw(0,0)
  to[short](-1,0);
  \draw(0,2)
  to[short](-1,2);
  \draw(4,0)
  to[short](5,0);
  \draw(4,2)
  to[short](5,2);
  \draw(1,1)
  node[scale=2.5]{$\circlearrowright$} ;
  \draw(1,1)
  node{$i_L$};
  \draw(3,1)
  node{$v_L$};
  \draw(2.5,1.4)
  node{$+$};
  \draw(2.5,0.6)
  node{$-$};
\end{circuitikz}
\end{figure}
O que se move com velocidade $v$?
\begin{itemize}
  \item À medida que o tempo passa, a corrente no primeiro indutor carrega o capitor seguinte
  \item A variação da tensão no capacitor provoca a variação da corrente no indutor seguinte
  \item Resultado:
    \begin{itemize}
      \item A perturbação elétrica se propaga com velocidade $v_f$ ao longo da linha de transmissão
    \end{itemize}
\end{itemize}
O que acontece quando andamos um comprimento de onda na linha de transmissão?
\begin{align*}
  v^+(z,t)= V^+ \cos( \omega t - \beta z ) = V^+ \cos (\beta z - wt) \\
  v^+ (z+ \lambda, t) = V^+ \cos( \beta z + \beta \lambda- \omega t )= V^+ \cos ( \beta z - \omega t + 2\pi )= V^+ \cos(\beta z - \omega t)
\end{align*}
Assim temos:
\begin{align*}
  \beta \lambda= 2 \pi \\
  \beta = \frac{2\pi}{\lambda}
\end{align*}
Onde $\beta$ é a constante de fase dada em $[rad/$unid.comprimento$]$ ou $[$unid.comprimento$]^{-1}$.

\subsection{Constante de propagação e atenuação}
Se temos uma linha com perdas, existe uma atenuanção da tensão/corrente ao longo do eixo $z$;
\begin{align*}
  V^+ (z) = V^+ e^{-\gamma z} = V^+ e^{-(\alpha +j\beta)z}\\
  V^+ (z) = V^+ e^{-\alpha z} e^{-j \beta z}\\
  v^+ (z,t) = V^+ e^{-\alpha z} cos(wt -\beta z)
\end{align*}
Onde:
\begin{align*}
  \alpha& \text{: constante de atenuanção: [neper/metro] ou [dB/metro]}\\ 
  \beta&\text{: constante de fase :}[rad/\text{metro] ou }m^-1  \\
  V^+ e^{-\alpha z}&\text{: indica uma amplitude decrescendo exponencialmente}
\end{align*}
\subsection{Linha de transmissão sem perdas}
Se lembrarmos que $\gamma = ZY$, podemos dizer:
\begin{align*}
\frac{ \gamma }{Z} = \sqrt{ \frac{Y}{Z}} =\sqrt{\frac{G+ j \omega C}{R+ j \omega L}}= \sqrt{\frac{j \omega C}{j \omega L}}= \sqrt{\frac{C}{L}}
\end{align*}
Invertendo a equação anterior, encontramos a impedância característica, $Z_0$ da linha:
\begin{equation*}
  Z_0= \sqrt{ \frac{R+j \omega L}{G+ j \omega C} } =\sqrt{ \frac{L}{C} } \text{[ohm]}
\end{equation*}
De $\gamma$:
\begin{align*}
  \gamma & = \sqrt{ZY} = \sqrt{(R + j \omega L)(G + j \omega C)} 
\end{align*}
Em uma linha sem perdas, $R=G=0$:
\begin{align*}
  \gamma &= \sqrt{(j \omega L)(j \omega C)}=\sqrt{-\omega^2 LC}=j\omega \sqrt{LC} = j \beta \\
  \beta &= \omega \sqrt{LC} = \frac{\omega}{V_{fase}} \quad [m^{-1}]
\end{align*}
Então em uma linha de transmissão sem perdas temos:
\begin{align*}
V(z)= V^+ e^{-j \beta z} + V^- e^{+j \beta z} \\
I(z)= \frac{V^+}{Z_0} e^{-j \beta z} - \frac{V^-}{Z_0} e^{+j \beta z}
\end{align*}
Podemos aproximar linhas, que operam em frequências maiores que $100kHz$,  em que $\omega L \gg R$ e  $\omega C \gg G$  como linhas sem perdas em que $Z_{0,af} \equiv Z_0 \simeq \sqrt{\frac{L}{C}}$. E linhas operando em frequências de aproximadente 1 $kHz$ em que $\omega L \ll R$ e $ \omega C \ll G$ como puramente resistivas, assim temos $z_{0,bf} \simeq \sqrt{\frac{R}{G}}$.

\subsection{Coeficiente de Reflexão}
\begin{align*}
  \tau (z) = \frac{V^-(z)}{v^+ (z)} = (\frac{V^-}{V^+})\frac{e^{\gamma z}}{e^{-\gamma z}} = \Gamma_0 \frac{e^{\gamma z}}{e^{-\gamma z}}= \Gamma_0 e^{2 \gamma z}
\end{align*}
Coeficiente de reflexão na carga, em $z=0$ ($z<0$ em direção ao gerador):
\begin{align*}
  \tau(z=0)= \frac{V^-}{V^+}\equiv \Gamma_0 = |\Gamma_0| e^{j \phi_r}\\
  \tau (z) = | \Gamma_0 | e^{2 \gamma z + j \phi_r}
\end{align*}
Para o caso sem perdas, isto é, $R=G=0$, $\gamma = j \beta$:
\begin{align*}
  \tau(z) = | \Gamma_0| e^{\left[j(2\beta z + \phi_r) \right]}
\end{align*}
\subsection{Impedância ao Longo de Linha sem perdas}
\begin{align*}
  Z(z)= \frac{V(z)}{I(z)}= \frac{V^+ (z) + V^- (z)}{I^+ (z) - I^- (z)}= Z_0 \frac{V^+ e^{-j\beta z} + V^- e^{j \beta z}}{V^+ e^{-j \beta z} - V^- e^{j \beta z}}\\
  Z(z)=Z_0 \frac{e^{-j \beta z} + (V^- / V^+) e^{j \beta z}}{e^{-j \beta z} - (V^- / V^+) e^{j \beta z}}= Z_0 \frac{e^-{j \beta z} - (V^-/ V^+) e^{j \beta z}}{e^{-j \beta z} - (V^-/ V^+)e^{j\beta z}}= Z_0 \frac{e^{-j \beta z} + \Gamma_0 e^{+ j \beta z} }{e^{- j \beta z} - \Gamma_0 e^{j\beta z}}\\
  Z(z)= Z_0 \frac{(\cos \beta z - j \sin \beta z)+ \Gamma_0(\cos \beta z + \sin \beta z)}{(\cos \beta z - j\sin \beta z)- \Gamma_0 (\cos \beta z + j \sin \beta z)}
\end{align*}
Dividindo por $\cos (\beta z)$ em ambos os lados:
\begin{align*}
  Z(z)= Z_0 \frac{( 1-j\tan \beta z )+ \Gamma_0 (1+ j \tan \beta z)}{(1- j \tan \beta z)-\Gamma_0 (1+j \tan \beta)}= Z_0 \frac{( 1+\Gamma_0 )+tau_0(1+j\tan \beta z)}{(1-j\tan \beta z)- \Gamma_0 (1+j\tan \beta z)}
\end{align*}
Dividindo por $(1-\Gamma_0)/(1+/tau_0)$,
\begin{align*}
  Z(z)=Z_0 \frac{ \left[ (1+\Gamma_0)/(1-\Gamma_0) \right]- j \tan (\beta z) }{1-j \left[ (1+\Gamma_0)/(1-\Gamma_0) \right] \tan(\beta z)}
\end{align*}
A expressão acima deve ser igual a impedância de carga em $z=0$, i.e.\,
\begin{align*}
  Z(z=0)= Z_0 \frac{1+\Gamma_0}{1-\Gamma_0}= Z_L
\end{align*}
De onde obtemos o coeficiente de reflexão na carga:
\begin{align*}
  \Gamma_0 = \frac{Z_L- Z_0}{Z_L + Z_0}
\end{align*}
Mas,
\begin{align*}
  \Gamma_0 = \frac{Z_l - Z_0}{Z_L + Z_0}
\end{align*}
E
\begin{align*}
  \frac{1+\Gamma_0}{1-\Gamma_0} = \frac{Z_L}{Z_0}
\end{align*}
Então:
\begin{align*}
  Z(z)= Z_0 \frac{Z_l - j Z_0 \tan (\beta z)}{Z_0 - j Z_L \tan(\beta z)} \quad z \leq 0
\end{align*}
\subsection{Funções Hiperbólicas: Algumas relações}
\begin{align*}
  \tanh (x \pm y) = \frac{\tanh (x) \pm \tanh y}{1 \pm \tanh x \tanh y}\\
  \tanh (jx)= j \tan x \\
  \tan (jx) = j \tanh x
\end{align*}
\subsection{Impedância ao Longo da linha com Perdas}
\begin{align*}
  Z(z)= \frac{V^+ (z) + V^- (z)}{I^+ (z) + I^- (z)}=Z_0 \frac{e^{-\gamma z}+\Gamma_0 e^{\gamma z}}{e^-{\gamma z}- \Gamma_0 e^{\gamma z}}
\end{align*}
Como $\sinh x= \frac{e^x - e^-x}{2}$ e $\cosh x = \frac{e^x + e^{-x}}{2}$, então, $e^{\pm kz}= \cosh (kz) \pm \sinh(kz)$, temos:
\begin{align*}
  Z(z)= Z_0 \frac{Z_L - Z_0 \tanh (\gamma z)}{Z_0 - Z_L \tanh(\gamma z)} \quad z \leq 0
\end{align*}
Com $\gamma = \alpha + j \beta$
\subsection{Potência e Potência média na Linha de Transmissão}
A potência instantanêa em qualquer ponto da linha de transmissão é dada por:
\begin{align*}
p(z,t) = v(z,t)i(z,t) \quad W
\end{align*}
A potência instantanêa transportada pela onda incidente é:
\begin{align*}
p^+ (z,t)= v^+ (z,t) i^+ (z,t) \quad W
\end{align*}
A potência transportada pela onda incidente é, portanto, 
\begin{align*}
  p_m^+ = \frac{1}{2} Re \left\{ V^+ (z) \left[ I^+ (z) \right]* \right\}\\
  p_m^+ = \frac{1}{2} Re \left\{ \left| V^+(z)\right|^2 \right\}= \frac{1}{2 }\frac{ \left| V^+ \right|^2 }{z_0} \quad W
\end{align*}
Se a LT não possui perdas, a potência média entregue ao gerador, na entrada da linha, deve ser a mesma entregue pela linha à carga:
\begin{align*}
  (p_m)_i = (p_m)_L \to \frac{1}{2} Re \left\{ V(-l) I^*(-l) \right\}= \frac{1}{2} Re \left\{ V(0) I^*(0) \right\}
\end{align*}
\subsection{Impedância ao longo da linha}
\begin{align*}
  Z(z)= Z_0 \frac{Z_L - jZ_0 tan(\beta z)}{Z_0 - j Z_L \tan (\beta z)} \quad z \leq 0 
\end{align*}
Ou, com $\gamma = \alpha + j \beta$ 
\begin{align*}
  Z(z)= Z_0 \frac{Z_L - Z_0 \tanh(\gamma z)}{Z_0 - Z_L \tanh (\gamma z)} \quad z \leq 0
\end{align*}
\subsection{Terminação: Curto-circuito}
Nesse caso, $Z_L = 0$.
\begin{align*}
  Z(z=-l)= Z_0 \frac{0+j Z_0 \tan \beta l}{Z_0 + j(0) \tan \beta l}= j Z_0 \tan(\beta L)\\
  Z(z= -l)= j Z_0 \tan (\beta l)
\end{align*}
Normalizando:
\begin{align*}
  Z(z=-l)/(Z_0)= j \tan (\beta l) = jX
\end{align*}
\subsection{Terminação: Circuito Aberto}
Nesse caso, $Z_L \to \infty$
\begin{align*}
  Z(z=-l) = Z_0 \frac{Z_L+ jZ_0 tan(\beta l)}{Z_0 + j\tan(\beta l)}= Z_0 \frac{1+j(Z_0/Z_l)\tan(\beta l)}{(Z_0/Z_L)+j \tan(\beta l)} = -j Z_0 \cot (\beta l)
\end{align*}
Normalizando:
\begin{align*}
  \frac{Z(z=-l)}{Z_0}= -j \cot (\beta l) = -j X
\end{align*}
\subsection{Onda Estacionária: Módulo}
Lembremos a solução para a equação de onda:
\begin{align*}
V(z)= V^+ e^{-j \beta z} + V^- e^{j\beta z}
\end{align*}
Usando o valor do coeficiente de reflexão, $\Gamma_0= \frac{V^-}{V^+}$, podemos escrever:
\begin{align*}
  V(z)= V^+ (e^{-j \beta z}+ \Gamma_0 e^{j\beta z})
\end{align*}
Assim podemos escrever:
\begin{align*}
|V(z)|^2 &= V(z)V(z)^*\\
&= \left[ V^+ \left( e^{-j\beta z} + \Gamma_0 e^{j\beta z} \right) \right] \left[ \left( V^+ \right)^* \left( e^{j\beta z} + \tau^*_0 e^{-j \beta z} \right) \right]\\
&= V^+ \left( V^+ \right)^* \left( e^{- j \beta z} + \Gamma_0 e^{j \beta z} \right) \left( e^{j \beta z } + \Gamma^*_0 e^{-j\beta z} \right) \\
&= \left| V^+ \right|^2 \left( 1+ \Gamma_0^* e^{-2j\beta z} + \Gamma_0 e^{j 2 \beta z} + \Gamma_0 \Gamma_0^* \right) \\
&= \left| V^+ \right|^2 \left( 1+ \Gamma_0^* e^{-j2\beta z} + \Gamma_0 e^{j2\beta z} + \left| \Gamma_0 \right|^2 \right)
\end{align*}
Mas sabemos que:
\begin{align*}
  \Gamma_0 = \left| \Gamma_0 \right|e^{j \phi_r}
\end{align*}
E consequentemente:
\begin{align*}
  \Gamma_0^* e^{-j2\beta z} + \Gamma_0 e^{j 2 \beta z} = \left| \Gamma_0 \right| e^{- j \phi_r}e^{-j2\beta z} + \left| \Gamma_0 \right|e^{j \phi_r}e^{j\beta z}=\\  \left| \Gamma_0 \right| \left[ e^{-j \left( 2 \beta z + \phi_r \right)} + e^{j \left( 2\beta z + \phi_r \right)} \right] = 2 \left| \Gamma_0 \right| \cos \left( 2 \beta z + \phi_r \right) 
\end{align*}\
Assim concluímos que:
\begin{align*}
  \left| V(z) \right|^2 = \left| V^+ \right|^2 \left[ 1 + \left| \Gamma_0 \right|^2 + 2 \left| \Gamma_0 \right|\cos \left( 2\beta z + \phi_r \right) \right]
\end{align*}
Máximo ocorre para $\cos(2 \beta z + \phi_r)=1$:
\begin{align*}
  2 \beta z_max + \phi_r = 2 n \pi , \quad n=0,1,2,3 \ldots \\
  \left| V(z) \right|^2_{max} = \left| V^+ \right|^2 \left( 1 + \left( \Gamma_0 \right)^2 + 2 \left| \Gamma_0 \right| \right)= \left| V^+ \right|^2 \left( 1 + \left| \Gamma_0 \right|^2 \right)
\end{align*}
Finalmente, obtemos:
\begin{align*}
  \left| V(z) \right|_{max} \equiv V_{max} = \left| V^+ \right| \left( 1+ \left| \Gamma_0 \right| \right)
\end{align*}
De forma análoga, o mínimo ocorre para $ \cos \left( 2 \beta z + \phi_r \right)= -1$:
\begin{align*}
  2 \beta z_{min} + \phi_r = \left( 2n +1 \right)\pi \quad , n=0,1,2,3 \ldots \\
  \left| V(z) \right|^2_{min} = \left| V^+ \right|^2 \left( 1+ \left| \Gamma_0 \right|^2 - 2 \left| \Gamma_0 \right| \right) = \left| V^+ \right|^2 \left( 1 - \left| \Gamma_0 \right|^2 \right)
\end{align*}
Assim obtemos:
\begin{align*}
  \left| V(z) \right|_{min} = \equiv V_{min} = \left| V^+ \right|(1- \left| \Gamma_0 \right|)
\end{align*}
\subsection{Relação da Onda Estacionária (ROE)}
\begin{align*}
  ROE = \frac{ \left| V(z)_{max} \right| }{ \left| V(z)_{min} \right| }= \frac{ \left| V^+ \right| \left( 1 + \left| \Gamma_0 \right| \right) }{ \left| V^+ \right| \left( 1- \left| \Gamma_0 \right| \right) } = \frac{ \left( 1+ \left| \Gamma_0 \right| \right) }{ \left( 1 - \left| \Gamma_0 \right| \right)}
\end{align*}
\begin{align*}
  ROE = \frac{ \left| I(z)_{max} \right| }{ \left| I(z)_{min} \right|} = \frac{ \left| I^+ \right| \left( 1+ \left| \Gamma_0 \right| \right) }{ \left| I^+ \right| \left( 1- \left| \Gamma_0 \right| \right) }= \frac{ \left( 1+ \left| \Gamma_0 \right|\right) }{1 - \left| \Gamma_0 \right|}
\end{align*}
Se $ \left| \Gamma_0 \right|=0$ temos $ROE=1$, o que significa um casamento de impedância. Caso $ \left| \Gamma_0 =1 \right|$ temos $ROE \to \infty$ o que significa reflexão total. E sempre temos $ROE \geq 1$.
\end{document}
