\documentclass[11pt,a4paper]{article}
\usepackage[portuguese]{babel}
\usepackage[utf8]{inputenc}
\usepackage{amsmath}
\usepackage{amsfonts}
\usepackage{enumitem}
\usepackage{booktabs}
\usepackage{circuitikz}
\begin{document}
\section{Linhas de Transmissão}
Para modelarmos uma linha de transmissão, podemos representá-la com um modelo simples utilizando modelos concentrados, em que representam a linha como 
uma junção de vários circuitos em série em um comprimento $\Delta Z$, assim como mostra a figura Figura~\ref{circ:1}.
\begin{figure}[htpb]
  \centering
  \begin{circuitikz}
    \draw(0,0)
    to[short] (1,0)
    to[R=$G\Delta z$](1,-3)
    to[short] (0,-3);
    \draw(1,0)
    to[short](2,0)
    node[label={[font=\footnotesize]above:A}]{}
    to[short,*-](3,0)
    to[C=$C \Delta z$](3,-3)
    to[short,-*](2,-3)
    to[short](1,-3);
    \draw(3,0)
    to[R=$R \Delta z$,i>^=$i_{(z,t)}$](5,0)
    to[short](5.5,0)
    to[L=$L \Delta z$](6.5,0)
    to[short](7,0)
    to[R=$G \Delta z$,i>^=$i_{G}$](7,-3)
    to[short](2,-3);
    \draw(6.5,0)
    to[short](7,0)
    to[short,-*](8,0)
    node[label={[font=\footnotesize]above:B}]{}
    to[short](9,0)
    to[short,i>^=$i_{z+\Delta z,t}$](10,0)
    ;
    \draw(9,0)
    to[C=$C \Delta z$,i>^=$i_{c}$](9,-3)
    to[short,-*](8,-3)
    to[short](7,-3) 
    ;
    \draw(9,-3)
    to[short](10,-3);
  \end{circuitikz}

  \caption{Circuito equivalente a linha de transmissão no pedaço de linha de comprimento $\Delta z$ com $R \Delta z$ representado as perdas do condutor em Ohms,  a condutância $G \Delta z$ representado as perdas do dielétrico em siemens, a indutância $L \Delta z$ do condutor em henrys e a capacitância $C \Delta z$ em farads.}
  \label{circ:1}
\end{figure}

Aplicando a lei de kirchoff das tensões no trecho $\Delta z$, obteremos:
\begin{align*}
  v(z,t)=i(z,t)R\Delta Z+\frac{\partial i (z,t)}{\partial t}L\Delta z+v(z+\Delta z,t)
\end{align*}
Dividino por $\Delta z$ e rearrajando:
\begin{align*}
  -  \frac{ v(z+\Delta,t)-v(z,t) }{\Delta z} = Ri(z,t)+L \frac{\partial i(z,t)}{\partial t}
\end{align*}
Mas sabemos que:
\begin{align*}
  \lim_{\Delta z \to 0}  \frac{ v(z+\Delta,t)-v(z,t) }{\Delta z}=\frac{\partial v(z,t)}{\partial z}
\end{align*}
Logo:
\begin{align}
  \label{eq:1}
  -\frac{\partial v(z,t)}{\partial z}=Ri(z,t)+L\frac{\partial i(z,t)}{\partial t}
\end{align}
\begin{figure}[hptb]
  \centering
  \begin{circuitikz}
    \draw(0,0)
    to[short,i>^=$i_{z,t}$,-*](2,0)
    node[label={[font=\footnotesize]above:B}]{}
    to[short,i>^=$i_{z+\Delta z,t}$](4,0)
    ;
    \draw(2,0)
    to[short,i>=$i_{G}_{(z+\Delta z,t)}+i_{C}_{(z+ \Delta z,t)}$](2,-2)
    ;
  \end{circuitikz}
  \caption{Correntes entrando e saindo o nó \emph{B}}
  \label{circ:2}
\end{figure}
Vamos agora determinar a outra equação equivalente a Equação~\ref{eq:1}, fazendo uso da equação de Kirchhoff nos nós da ponto \emph{B}.

\begin{align*}
  i(z,t) =i_{G}(z+\Delta z,t)+i_{C}(z+\Delta z,t)+i(z+\Delta z,t)
\end{align*}
Mas sabemos que:
\begin{align*}
  i_{G} (z+\Delta Z,t) &=v(z+ \Delta z,t)G \Delta Z \\
  i_{C}(z+ \Delta z,t) &= \frac{\partial v(z+ \Delta z,t)}{\partial t}C \Delta z
\end{align*}

Que substituindo resulta em:
\begin{align*}
  i(z,t) &= v(z+\Delta z, t)G\Delta z + \frac{\partial v(z+\Delta z,t)}{\partial t} C \Delta z + i(z+ \Delta z, t)
\end{align*}
Reescrevendo e divindo por $\Delta z$, 
\begin{align*}
  -\frac{i(z+\Delta z,t)-i(z,t)}{\Delta z} &= v(z+\Delta z, t)G + \frac{\partial v(z+\Delta z, t)}{\partial t}C
\end{align*}
Observar que:
\begin{equation*}
\lim_{ \Delta z \to 0} \left[ \frac{i(z+\Delta z,t) -i(z,t)}{\Delta z} \right] = \frac{ \partial i(z,t)}{\partial z}
\end{equation*}
Se temos que,
\begin{equation*}
  v(z+ \Delta z)  = v(z)+ \frac{\partial v (z)}{\partial z}\Delta z
\end{equation*}
Podemos concluir que:
\begin{align*}
  \frac{i(z+\Delta z,t) -i(z,t)}{\Delta z} &= v(z+\Delta z, t)G + \frac{\partial v(z+\Delta z,t)}{\partial t}C\\
  &= G \left[v(z,t) + \frac{\partial v(z,t)}{\partial z} \Delta z \right]+ C \left[ \frac{\partial v(z,t)}{\partial t} + \frac{\partial^2 v(z,t)}{\partial t \partial z}\Delta z \right] \\
  &= \left( G+C \frac{\partial}{\partial t}\right) \left[ v(z,t)+ \frac{ \partial v (z,t) }{\partial z} \Delta z \right]
\end{align*}
Que para $\Delta z \to 0$ se torna:
\begin{equation}
  -\frac{\partial i(z,t)}{\partial z} = Gv(z,t)+ C \frac{\partial v(z,t)}{\partial t}
\end{equation}
Dessa forma obtivemos as duas equações fundamentais da linha de transmissão

\end{document}
